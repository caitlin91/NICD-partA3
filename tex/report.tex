\documentclass{scrartcl}

\usepackage{hyperref}

\author{Caitlin Halfacre}
\date{\today}
\title{NICD Data Scientist - Task A3}

\begin{document}
\maketitle
I have chosen to use a Bayesian regression model to fit the clients desire for a model that both describes more of the variance in the data than their existing model and maintains predictive accuracy. While Bayesian modelling is computationally and technically more complex than a standard linear regression model, the approach to prediction and inference is more intuitive, that is, in line with how the human brain processes probability, and enables the option to present percentage likelihoods in explanations of valuations. In my analysis of the model I present visualisations and phrasing that could be used to explain valuation decisions to clients, regulators, and internal stakeholders.

\section{Model Design}
\subsection{Priors}
The priors are set using a very weakly informative understanding of the average and maximum likely values for the data\footnote{for example, \url{https://www.chroniclelive.co.uk/news/north-east-news/most-expensive-north-east-houses-30988236} the most expensive house sold in the North East in 2024}. As discussed in section \ref{limitations} these can be improved with greater research of the property market and discussion with the client.


\subsection{Model Selection}
Model selection was performed bby building all possible models with the given predictors
highest bayes\_R2 - model explains more variance.

highest elpd\_diff in your table - best expected predictive accuracy based on your negated version.

\section{Model Analysis}
x

\section{Limitations} \label{limitations}
One of the biggest advantages of Bayesian modelling is the ability to constrain a model using domain specific knowledge. This can be as little as knowing that a price cannot be negative, but is far more uiseful if more information can be added 
empty data
domain specific knowledge
types of data - garden


\end{document}